\chapter*{Preface}
\addcontentsline{toc}{chapter}{Preface}

The goal of this book is to cover all the material that a competent
multiagent practitioner or researcher should be familiar with. Of
course, since this is a relatively new field the list of required
material is still growing and there is some uncertainty as to what are
the most important ideas. I have chosen to concentrate on the
theoretical aspects of multiagent systems since these ideas have been
around for a long time and are important for a wide variety of
applications. I have stayed away from technological issues because
these are evolving very fast.  Also, it would require another textbook
to describe all the distributed programming tools available. A reader
interested in the latest multiagent technologies should visit the
website \url{www.multiagent.com}.

The book is linked to a large number of sample NetLogo programs
\cite{netlogo}.  These programs are meant to be used by the reader as
an aid in understanding emergent decentralized behaviors. It is hard
for people, especially those new to distributed systems, to fully
grasp the order that can arise from seemingly chaotic simple
interactions. As Resnick notices:\mc{Resnick points to examples such
  as surveys of 8--15 year old kids, half of which believe that the
  government sets all prices and salaries.}
\begin{quotation}
  \emph{``But even as the influence of decentralized ideas grows, there is a
  deep-seated resistance to such ideas. At some deep level, people
  seem to have strong attachments to centralized ways of
  thinking. When people see patterns in the world (like a flock of
  birds), they often assume that there is some type of centralized
  control (a leader of the flock). According to this way of thinking,
  a pattern can exist only if someone (or something) creates and
  orchestrates the pattern. Everything must have a single cause, and
  ultimate controlling factor. The continuing resistance to
  evolutionary theories is an example: many people still insist that
  someone or something must have explicitly designed the complex,
  orderly structures that we call Life.''} \cite{resnick94a}
\end{quotation}\mc{Resnick created StarLogo in order to teach the
  \emph{decentralized} mindset. Wilensky, one of his students, later
  extended StarLogo and created NetLogo.}


The reader is assumed to be familiar with basic Artificial
Intelligence techniques \cite{russell03a}. The reader should also be
comfortable with mathematical notation and basic computer science
algorithms. The book is written for a graduate or advanced
undergraduate audience. I also recommend
\cite{mas-colell95a,osborne99a} as reference books.


\section{Usage}
\label{sec:usage}

If you are using the \textsc{pdf} version of this document you can
click on any of the citations and your \textsc{pdf} reader will take
you to the appropriate place in the bibliography. From there you can
click on the title of the paper and your web browser will take you to
a page with a full description of the paper. You can also get the full
paper if you have the user-name and password I provide in class. The
password is only available to my students due to licensing
restrictions on some of the papers.

Whenever you see an icon such as the one on this margin
\netlogo{ABTgc} it means that we have NetLogo implementation of a
relevant problem. If you are using the \textsc{pdf} version of this
document you can just click on the name and your browser will take you
to the appropriate applet. Otherwise, the \textsc{url} is formed by
pre-pending \url{http://jmvidal.cse.sc.edu/netlogomas/} to the name
and appending \url{.html} at the end. So the \textsc{url} for this
icon is \url{http://jmvidal.cse.sc.edu/netlogomas/ABTgc.html}.


\section{Acknowledgments}

I would like to thank the students at the University of South Carolina
who have provided much needed feedback on all revisions of this book.
Specifically, I thank Jimmy Cleveland, Jiangbo Dang, Huang Jingshan,
and Alicia Ruvinsky. I am also especially grateful to faculty members
from other Universities who have used this book in their classes and
provided me with invaluable feedback. Specifically I thank Ram\'{o}n
F. Brena, Muaz Niazi, and Iyad Rahwan.



%%% Local Variables: 
%%% mode: latex
%%% TeX-master: "~/wp/mas/mas"
%%% TeX-command-default: "PDFlatex"
%%% End:

